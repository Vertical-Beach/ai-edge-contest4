\section{ハードウェア最適化}
Xilinx社から提供されるDPU IPコアは,画素や入出力チャネルに対する並列数が異なる複数の種類のIPコアが提供されている.
より並列数の高いIPコアを使用することで処理性能が向上するが,回路規模および消費電力が増加する.
また,DPUコアの一部のレイヤーのサポートを無効にすることでリソース使用率を低減することができる.

コンテストの評価ボードであるUltra96V2に搭載可能なDPUコアとしてB2304を採用した.
デフォルトで有効になっているDepthWiseConvolutionおよびPool Averageのレイヤーは今回設計したモデルでは必要ないため無効にした.

さらに,DPUの動作周波数を高めることにより,DPUにおける推論実行時間を短縮することができる.
論理合成のストラテジをFlow\_AreaOptimized\_high,配置配線のストラテジをperformance\_ExtraTimingOptに変更することで動作周波数を150MHz/300MHzから200MHz/400MHz\footnote{DPUコアはベース周波数に加えてその2倍の周波数のクロックをDSPに接続するため,動作周波数はこのような表記とした.}に高めてもタイミング制約を満たし,FPGAビットストリームの生成を行うことができた.

表\ref{runtimetable}に動作周波数と入力画像サイズごとのDPUにおける推論実行時間・およびスコアを示す.
DPUの推論時間は入力画像サイズに概ね比例し,動作周波数を向上させることによって推論処理が約1.2倍高速化されることがわかる.
\begin{table}[h]
    \label{runtimetable}
    \caption{各入力画像サイズ・動作周波数における推論時間とスコア}
    \begin{center}
        \begin{tabular}{lllll}
            Image Size & \multicolumn{2}{c}{DPUTask {[}ms{]}}                            & \multicolumn{1}{c}{Score} &  \\
                & \multicolumn{1}{r}{150/300MHz} & \multicolumn{1}{r}{200/400MHz} &                           &  \\ \hline
        256*512 & 35                             & 30                             & 0.539                     &  \\ \hline
        320*640 & 60                             & 55                             & 0.579                     &  \\ \hline
        352*704 & 72                             & 65                             & 0.593                     &  \\ \hline
        384*768 & 81                             & 70                             & 0.608                     &  \\ \hline
        480*960 & 128                            & 110                            & 0.616                     & 
        \end{tabular}
    \end{center}
\end{table}
